\RequirePackage[l2tabu,orthodox]{nag}  % warn about common LaTeX pitfalls
\RequirePackage[ascii]{inputenc}  % input is 7-bit ASCII
\RequirePackage{fixltx2e}  % fix LaTeX2e kernel bugs

\documentclass[11pt,twoside]{article}
\usepackage{tikz}
\usepackage{tkz-graph}
\usepackage{pgfplots}
\usepackage{color}
\usepackage{calc}  % arithmetic in length parameters
\usepackage{enumitem}  % more control over list formatting
\usepackage{fancyhdr}  % simpler headers and footers
\usepackage[margin=1in]{geometry}  % page layout
\usepackage{lastpage}  % for last page number
\usepackage{relsize}  % easier font size changes
\usepackage[normalem]{ulem}  % smarter underlining
\usepackage{url}  % verb-like typesetting of URLs
%\usepackage{xfrac}  % nicer looking simple fractions for text and math
\usepackage{amsmath}
\usepackage{amssymb}

% Set up fonts.
\usepackage[T1]{fontenc}  % use true 8-bit fonts
\usepackage{slantsc}  % allow slanted small-caps
\usepackage{microtype}  % perform various font optimizations
% Use Palatino-based monospace instead of kpfonts' default.
%\usepackage{newpxtext}
\ttfamily
\DeclareFontShape{T1}{\ttdefault}{m}{scsl}{<->ssub*\ttdefault/m/sc}{}
\DeclareFontShape{T1}{\ttdefault}{b}{scsl}{<->ssub*\ttdefault/b/sc}{}
% "Kepler" fonts.
\usepackage[nott,notextcomp]{kpfonts}
% Use curvier Latin Modern brackets instead of kpfonts' glyphs.
\DeclareSymbolFont{lmsymb}     {OMS}{lmsy}{m}{n}
\DeclareSymbolFont{lmlargesymb}{OMX}{lmex}{m}{n}
\DeclareMathDelimiter{\rbrace}{\mathclose}{lmsymb}{"67}{lmlargesymb}{"09}
\DeclareMathDelimiter{\lbrace}{\mathopen}{lmsymb}{"66}{lmlargesymb}{"08}

% Page layout: stretch text to fill up page.
\addtolength\footskip{.25\headheight}
\flushbottom


% Headings.
\pagestyle{fancy}
\let\headrule\empty
\let\footrule\empty
\lhead{CSC\,438\,H1}
\chead{\large\scshape Assignment \#\ 2}
\rhead{\scshape Fall 2015}
\cfoot{}
\rfoot{\scshape page \thepage\space of \pageref{LastPage}}
\begin{document}
\[Rui \ \ Ji \ (1000340918)\]
\begin{enumerate}
\item
Proof:
	\begin{itemize}[label = {}]
		\item Because the $Restriction$ for this rule, the variable $b$ cannot occur in $\Gamma$ or $\Delta$. Hence, it suffices to verify that
				\[\forall x (\bigwedge \Gamma, A(x) \supset  \bigvee \Delta) \models (\bigwedge \Gamma,  \exists A(x))\supset  \bigvee \Delta\]
		\item to see that this logic consequence holds, suppose $\mathcal {M}$ satisfies the left hand side, then  $\mathcal {M}$ satisfies $\forall$ A, hence, 
		 $\mathcal {M}$ satisfies $\existl$ A which means,  $\mathcal {M}$  satisfies that right hand side.
	\end{itemize}
\item	We need the following LK Equality Axioms:\\
	\begin{itemize}[label = {}]
	\item EL1: $\rightarrow a = a$ 
	\item EL4: $a=a, \ b + 0 = b  \rightarrow a+(b+0) = a+b$
	\item
	\end{itemize}
	Here is the LK proof:
	\[ EL1\ \ \ \ \ \ \ \ \ \ \ \ \ \  \ \ \ \ EL4\]
	\[ \ \rule{3cm}{0.4pt} \ \ \ \rule{3cm}{0.4pt}\] 
	\[ \rightarrow a = a \ \ \ \ \ \ a=a, \ b + 0 = b  \rightarrow a+(b+0) = a+b\]
	\[ \ \ \ \ \ \ \ \ \ \ \ \  \ \ \  \rule{1.5cm}{0.4pt} \ \ \ \rule{6cm}{0.4pt} \ cut \ rule\] 
	\[b+0 = b \rightarrow a+(b+0) = a+b \]
	\[\ \ \ \ \ \ \ \ \ \ \ \  \ \ \  \rule{8cm}{0.4pt} \forall \ left\]
	\[ \forall x (x+0 = x) \rightarrow a+(b+0) = a+b)\] 
	\[\ \ \ \ \ \ \ \ \ \ \ \  \ \ \  \rule{8cm}{0.4pt} \forall \ right\]
	\[ \forall x (x+0 = x) \rightarrow  \forall y \ a+(y+0) = a + y\] 
	\[\ \ \ \ \ \ \ \ \ \ \ \  \ \ \  \rule{8cm}{0.4pt} \forall \ right\]
	\[ \forall x (x+0 = x) \rightarrow  \forall x \forall y \ x+(y+0) = x + y \] 
\item Proof:
	\begin{itemize}[label = {}]
	\item Using Compactness, we know that if $A$ is a logic consequence of $\Gamma$ then $A$ is a logic consequence of some finite subset $\Gamma$. Hence, it is surfficent to show that every finite subset of $\Gamma$ has a finite model.
	\item Suppose $S' \subset S$ is finite. Then there is some natural number $n$ such that $S'$ has no constant $c_i$ with $i > n$.
	\item Let $\matcal{M}$ be a model consist of a cycle of $n$ nodes. Let the constants $c_1, c_2, . . . , c_n$ to be nodes in the cycle and $P^M(c_i, c_j)$ means there is a edge between nodes $c_i$ and $c_j$. Clearly,  $\matcal{M}$ is a finite model for $S'$. 
	\item Hence,   every finite subset of $\Gamma$ has a finite model which means  $A$ has a finite model.
	\item 
	\end{itemize}
\item (a)
\begin{itemize}[label = {}]
\item let $P_1, P_2, ... P_n$ be the predicate symbols in A. and we know they all unary symbols.
\item Suppose A is satisfiable, then let $\mathcal{M}$ be a model for A over universe M, and let $\sigma$ be the associated object assignment, then we have $\mathcal{M}  \models A[\sigma]$.
\item Now define a equivalence relation $\sim$ on M:
	\[ \forall a, b \in M \ \ ( a \sim b \iff P_i^M[a] = P_i^M[b]) \ i \leq i \leq n\]
	Clearly, $\sim$ is a  equivalence relation.
\item Then every equivalence class $[x]$ is  a n-tuple $(P_1^M(a), P_2^M(a),. . . , P_n^M(a)$). Since $P_i^M$ is a boolean predicate then there are t most  $2^n$ such n-tuples; hence,  there are at most $2^n$ equivalence classes in M.
\item Now define a structure $\mathcal M'$ with a universe M' consisting all classes in M.
\[M' = \{ [ x ] \ | \ x \in M \}\]
\[P_i^{M'}([x]) = P_i^M(x)}\]
\item and the associated object assignment $\sigma'$
\[\sigma'([x]) = [\sigma(x)] \]
\item Then based on $Lemma \ 2$: $\mathcal{M'}  \models A[\sigma']$ iff $\mathcal{M}  \models A[\sigma]$, on page 45 we know that $\mathcal{M'}  \models A[\sigma']$.
\end{itemize}
(b)
\begin{itemize}[label = {}]
\item From (a) we know that for any model $\mathcal{M}$  associated object assignment $\sigma$ for A, we can choose an equivalent model with a finite universe which has at most $2^n$ elements. Hence,  A is valid iff it is satisfied by every structure with at most $2^n$ elements in its universe.
\item Hence,  we processing formula by replacing universal quantified sub-formulas with a conjunction of the sub-formulas over $2^n$ distinct elements and replacing existentially- quantified sub-formulas with a disjunction of the sub-formulas over $2^n$ distinct elements. 
\item We know that there are $2^n$ possible truth assignments for all $n$ predicates in the formula. Then we can just substitute all possible truth assignments and the the formula is valid iff , for all substitutions, it to true.
\item Since the number of the all possible truth assignments is bounded; hence, the algorithm will always halts.
\end{itemize}
\end{enumerate}
\end{document}
