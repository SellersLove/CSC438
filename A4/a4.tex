\RequirePackage[l2tabu,orthodox]{nag}  % warn about common LaTeX pitfalls
\RequirePackage[ascii]{inputenc}  % input is 7-bit ASCII
\RequirePackage{fixltx2e}  % fix LaTeX2e kernel bugs

\documentclass[11pt,twoside]{article}
\usepackage{color}
\usepackage{amsthm}
\usepackage{amsmath}
\usepackage{setspace}
\makeatletter
\newcommand{\dotminus}{\mathbin{\text{\@dotminus}}}

\newcommand{\@dotminus}{%
  \ooalign{\hidewidth\raise1ex\hbox{.}\hidewidth\cr$\m@th-$\cr}%
}
\makeatother
\usepackage{graphicx}
\graphicspath{ {image/} }
\usepackage{calc}  % arithmetic in length parameters
\usepackage{enumitem}  % more control over list formatting
\usepackage{fancyhdr}  % simpler headers and footers
\usepackage[margin=1in]{geometry}  % page layout
\usepackage{lastpage}  % for last page number
\usepackage{relsize}  % easier font size changes
\usepackage[normalem]{ulem}  % smarter underlining
\usepackage{url}  % verb-like typesetting of URLs
\usepackage{xfrac}  % nicer looking simple fractions for text and math
\usepackage{longtable}
\usepackage{tikz}
\usepackage{array}
\usepackage{tikz-timing}
\usetikzlibrary{arrows, shapes, backgrounds,fit}
\usepackage{tkz-graph}
% Set up fonts.
\usepackage[T1]{fontenc}  % use true 8-bit fonts
\usepackage{slantsc}  % allow slanted small-caps
\usepackage{microtype}  % perform various font optimizations
% Use Palatino-based monospace instead of kpfonts' default.
%\usepackage{newpxtext}
\ttfamily
\DeclareFontShape{T1}{\ttdefault}{m}{scsl}{<->ssub*\ttdefault/m/sc}{}
\DeclareFontShape{T1}{\ttdefault}{b}{scsl}{<->ssub*\ttdefault/b/sc}{}
% "Kepler" fonts.
\usepackage[nott,notextcomp]{kpfonts}
% Use curvier Latin Modern brackets instead of kpfonts' glyphs.
\DeclareSymbolFont{lmsymb}     {OMS}{lmsy}{m}{n}
\DeclareSymbolFont{lmlargesymb}{OMX}{lmex}{m}{n}
\DeclareMathDelimiter{\rbrace}{\mathclose}{lmsymb}{"67}{lmlargesymb}{"09}
\DeclareMathDelimiter{\lbrace}{\mathopen}{lmsymb}{"66}{lmlargesymb}{"08}

% Page layout: stretch text to fill up page.
\addtolength\footskip{.25\headheight}
\flushbottom

% Common list settings.

% Common macros.
%%  Common macros for the course CSC263H1 at the University of Toronto.
%%
%%  Copyright (c) 2014 Francois Pitt <fpitt@cs.utoronto.ca>
%%  last updated at 17:53 (EDT) on Sun 19 Oct 2014
%%
%%  CC BY-SA 4.0
%%  This work (the current file named 'macros-263.tex') is licensed under
%%  the Creative Commons Attribution-ShareAlike 4.0 International License.
%%  To view a copy of this license, visit
%%      http://creativecommons.org/licenses/by-sa/4.0/
%%  or send a letter to: Creative Commons, 444 Castro Street, Suite 900,
%%  Mountain View, California, 94041, USA.
%%  This is a human-readable summary of (and not a substitute for) the
%%  license.
%%  You are free to:
%%      Share -- copy and redistribute the material in any medium or format
%%      Adapt -- remix, transform, and build upon the material for any
%%          purpose, even commercially.
%%      The licensor cannot revoke these freedoms as long as you follow the
%%          license terms.
%%  Under the following terms:
%%      Attribution -- You must give appropriate credit, provide a link to
%%          the license, and indicate if changes were made. You may do so in
%%          any reasonable manner, but not in any way that suggests the
%%          licensor endorses you or your use.
%%      ShareAlike -- If you remix, transform, or build upon the material,
%%          you must distribute your contributions under the same license as
%%          the original.
%%      No additional restrictions -- You may not apply legal terms or
%%          technological measures that legally restrict others from doing
%%          anything the license permits.
%%  Notices:
%%      You do not have to comply with the license for elements of the
%%      material in the public domain or where your use is permitted by an
%%      applicable exception or limitation.
%%      No warranties are given. The license may not give you all of the
%%      permissions necessary for your intended use. For example, other
%%      rights such as publicity, privacy, or moral rights may limit how you
%%      use the material.

% Redefine \today in the style "DD Month YYYY".
\renewcommand*\today
 {\number\day\space\ifcase\month\or January\or February\or March\or
  April\or May\or June\or July\or August\or September\or October\or
  November\or December\fi\space\number\year}

% TeX trick so that math symbols are bold when text is.
\let\seiresfb\bfseries\def\bfseries{\boldmath\seiresfb}
\let\seiresdm\mdseries\def\mdseries{\unboldmath\seiresdm}

% Centered version of \llap and \rlap.
\providecommand*\clap[1]{\hbox to 0pt{\hss#1\hss}}

% Spacing macros with default argument.
\providecommand*\vfillstretch[1][1]{\vspace*{\stretch{#1}}}
\providecommand*\hfillstretch[1][1]{\hspace*{\stretch{#1}}}

% Abbreviations of latin phrases.
\let\latinabb\empty  % no special formatting
\providecommand*\ie{\latinabb{i.e.}}
\providecommand*\eg{\latinabb{e.g.}}
\providecommand*\etc{\latinabb{etc}}
\providecommand*\vs{\latinabb{vs}}

% General fonts and characters.
\let\longemph\textsl  % alternate way to emphasize
\let\strong\textbf  % strong emphasis
\let\code\texttt  % code fragments
\let\var\textsl  % multi-letter variables
\let\pred\mathbf  % predicates
\providecommand*\stup{\textsuperscript{st}}
\providecommand*\ndup{\textsuperscript{nd}}
\providecommand*\rdup{\textsuperscript{rd}}
\providecommand*\thup{\textsuperscript{th}}
\newcommand*\bbmath[1]{\ensuremath{\mathbb{#1}}}  % requires amssymb
\providecommand*\N{\bbmath{N}}  % requires amssymb
\providecommand*\Z{\bbmath{Z}}  % requires amssymb
\providecommand*\Q{\bbmath{Q}}  % requires amssymb
\providecommand*\R{\bbmath{R}}  % requires amssymb
\providecommand*\bigOh{\mathcal{O}}
\providecommand*\onehalf{\ensuremath{{}^1\!\!/\!_2}}
\providecommand*\letlinebreak{\penalty\exhyphenpenalty}
\providecommand*\halfthinspace{\kern.083333em}
\providecommand*\dash
   {\letlinebreak\halfthinspace---\halfthinspace\letlinebreak}
\let\per\slash  % for convenience

% For defined terms -- requires package ulem.
\newcommand*\defn[1]{\uline{\textit{#1}}}
\setlength{\ULdepth}{.2ex}

% General math macros.
\newcommand*\hiderel[1]{\mathrel{\hphantom{#1}}}
\providecommand*\comp[1]{\overline{#1}}  % set complement
\providecommand*\emptystr{\varepsilon}  % empty string
\providecommand*\lxor{\mathbin\oplus}  % exclusive or
\providecommand*\cat{\cdot}  % string concatenation
% Floors and ceilings, with optional size-changing command, e.g.,
% \floor[\Big]{x/2} becomes '\Big\lfloor{x/2}\Big\rfloor'.
\providecommand*\floor[2][]{{#1\lfloor}{#2}{#1\rfloor}}
\providecommand*\ceil[2][]{{#1\lceil}{#2}{#1\rceil}}
\providecommand*\lcm{\operatorname{lcm}}  % requires amsmath
\providecommand*\size{\operatorname{size}}  % requires amsmath
\providecommand*\len{\operatorname{len}}  % requires amsmath
\providecommand*\divides{\mathrel{|}}

% Redefined symbols from amssymb.
\renewcommand*\emptyset{\varnothing}  % rounder than default
\let\bigiff\iff
\renewcommand*\iff{\mathrel{\Leftrightarrow}}  % smaller than default
\let\bigimplies\implies
\renewcommand*\implies{\mathrel{\Rightarrow}}  % smaller than default
\renewcommand*\ge{\geqslant}  % with slanted line under the > sign
\renewcommand*\le{\leqslant}  % with slanted line under the < sign

% For algorithms, using either CLRS-style or Python-style pseudocode.
\let\ADT\textsc  % ADT names
\let\proc\textsc  % function names
\let\const\textsc  % constants (True, False, etc.)
\let\kw\textbf  % keywords (if, while, etc.)
\providecommand*\comm[1]{\textsl{\#\space#1}}  % comments
\providecommand*\opgets[1]{\mathrel{#1=}}
\providecommand*\eq{\mathrel{==}}
\providecommand*\True{\const{True}}
\providecommand*\False{\const{False}}
\providecommand*\None{\const{None}}
\providecommand*\cmod{\mathbin{\%}}
\providecommand*\nil{\const{nil}}

% Checkboxes and checklists.
% \checkmark already defined in amssymb
\makeatletter
\@ifpackageloaded{kpfonts}
   {\newcommand*\exmark{$\times$}}
   {\newcommand*\exmark{{\boldmath$\times$}}}
\makeatother
\newlength\exmarksize
\newlength\exmarkdepth
\newcommand*\checkbox[1][]
   {\settoheight\exmarksize{\exmark}
    \settodepth\exmarkdepth{\exmark}
    \addtolength\exmarksize{-\exmarkdepth}
    {\setlength\fboxrule{.1ex}\setlength\fboxsep{.1ex}%
    \fbox{\rule{0pt}{\exmarksize}\makebox[\exmarksize]{\smash{#1}}}}}
\newcommand*\checkedbox{\checkbox[\raisebox{.1ex}{\kern.2em$\checkmark$}]}
\newcommand*\exedbox{\checkbox[\exmark]}
\newenvironment*{checklist}{\begin{list}{\checkbox}{}}{\end{list}}
\newcommand*\checkeditem{\item[\checkedbox]}
\newcommand*\exeditem{\item[\exedbox]}
\newcommand*\boxeditem[1][]{\item[{\checkbox[#1]}]}

% Macros for drawing "underline" rules and half-boxes.
\providecommand*\urule[2][.5pt]{\rule[-.4ex]{#2}{#1}} % "underline" rule
% Underlined text box (with optional positioning argument).
\providecommand*\ubox[3][c]{\rlap{\urule{#2}}\makebox[#2][#1]{#3}}
% Underlined text box with caption (and optional caption positioning).
\providecommand*\capbox[5][c]{\ifx\empty#2\empty
    \else\rlap{\raisebox{-\baselineskip}{\makebox[#3][#1]{#2}}}\fi
    \ubox[#4]{#3}{#5}}

% For student numbers.
\providecommand*\hb{\urule{1.2em}}      % horizontal bar
\providecommand*\vb{\urule[1ex]{.5pt}}  % vertical bar
\providecommand*\studentnumberboxes     % now 10-digits long!
   {\vb\hb\vb\hb\vb\hb\vb\hb\vb\hb\vb\hb\vb\hb\vb\hb\vb\hb\vb\hb\vb}
\newlength\numberboxwidth
\settowidth\numberboxwidth{\studentnumberboxes}

% Command to format sample solutions for tutorials.
\newcommand\samplesolution[1]
   {\ifsolutions\begin{quote}\sffamily{#1}\end{quote}\fi}

% Marking-related macros.
\providecommand*\markitem[2][]
   {\item{\bfseries#2}\ifx\empty#1\empty\else
    \space[$#1$ mark\ifnum#1>1 s\fi]\fi:\quad\ignorespaces}
\providecommand*\errorcode[2][]
   {\item{\bfseries error code #2}\ifx\empty#1\empty\else
    \space[#1]\fi:\quad\ignorespaces}
\providecommand*\commonerror[1][]
   {\item{\bfseries common error}\ifx\empty#1\empty\else
    \space[#1]\fi:\quad\ignorespaces}
\makeatletter
\providecommand*\heading[1]%
   {\@startsection{heading}{9}{0pt}{-.75ex plus -1.5ex minus -.5ex}
    {-.5em plus -.5em minus -.25em}{\normalfont}*{\textsc{#1}}\mbox{}}
\makeatother
\newcommand*\st{\mathrel{|}}  % "such that" for set extension

% Headings.
\pagestyle{fancy}
\let\headrule\empty
\let\footrule\empty
\lhead{CSC\,418\,H1}
\chead{\large\scshape Assignment \#\,4}
\rhead{\scshape Fall 2015}
\lfoot{\scshape Dept.\@ of Computer Science, University of Toronto,
       St.~George Campus}
\cfoot{}
\rfoot{\scshape page \thepage\space of \pageref{LastPage}}


\begin{document}
\begin{enumerate}[leftmargin=0pt]
% question 1
\item 
	\proc{Proof:}
	\begin{itemize}[label ={}]
		\item $\Rightarrow$ Suppose $f(\vec x)$ is a $n$-$ary$ computable function.  Then there exist some program $\{e\}$ computes $f$,  by KNFT we get,
				\[f(\vec x) = \{e\}(\vec x) = U(\mu yT_n(e,\vec x,y))\]
		\item Let $graph(f) = \{(\vec x, y) \ | \ y=f(\vec x)\}$,
		then we can get that, 
				\[ (\vec x, y)= (x_1,x_2,...,x_n, y) \in graph(f) \iff \exists z(T_n(e,\vec x,z) \wedge U(z)=y) \]
		\item Hence, let $R(x_1,x_2,...,x_n, y ,z) = T_n(e,x_1,x_2,...,x_n, y,z) \wedge U(z)=y)$. Clearly $R$ is a recursive relation, then $graph(f)$ is r.e. 
		\item
		\item $\Leftarrow$  Before formalizing the argument for `if' direction,  let's first give an informal algorithm for computing $f$ from an enumeration of the tuples in $graph(f)$  
		\item algorithm (for computing $f(\vec x)$):
		\item \begin{itemize}[label={}]
			\item  suppose $(a_1, a_2....)$ is a enumeration, where $a_i$ is a $n+1$ tuple.
			\item for $i = 1...\infty$
			\item \hskip2pc  if $a_i = (x_1,x_2,...,x_n, y)$ where $  (x_1,x_2,...,x_n) = \vec x$
			\item \hskip4pc 	\proc {output} $y$
			\item \hskip2pc end if
			\item end for
			\end{itemize}
		
		\item Now let's formalize the argument. Suppose $graph(f) = \{(\vec x, y) \ | \ y=f(\vec x)\}$ is r.e., where $f$ is a $n-ary$ function and we want to show that $f(\vec x)$ is computable.  
		\item Since graph(f) is r.e., then there exist some $n+2$-$ary$ recursive relation $R(\vec x ,z )$such that,
			 \[(x_1,x_2,...,x_n, y) \in graph(f) \iff \exists z \ R(x_1,x_2,...,x_n, y, z) \]
		\item Then clearly $f(x_1,x_2,...,x_n) = \mu y (\exists z \ R(x_1,x_2,...,x_n, y, z))$, by Church-Turing Thesis, $f(x_1,x_2,...,x_n)$ is computable.
	\end{itemize}
\qed
% question 2
\item $\proc {Claim:}$ $A$ is neither recursive nor r.e., $ A^c$ is r.e but not recursive.
	\begin{itemize}[label = {}]
	% $A^c not  r.e$
	\item First we show that $A$ is not r.e. Notice, it suffices to show that
			\[K^c \leq_m A\] 
		Thus we want a total computable function $f(x)$ such that
			\[x \in K^c \iff f(x) \in A \]
		i.e we want
			\[ \{x\}_1(x) = \infty \iff range(\{f(x)\}_1) \subseteq ODD\]
		We can define $f(x)$ implicitly using the S-m-n Theorem as follows:
			\begin{gather*}
			\{f(x)\}_1(y) = 
				\begin{cases}
				0\cdot\{x\}_1(x)  & \text{if } y \neq 1\\
				1  & \text{if } y = 1
				\end{cases}
			\end{gather*}
		\item Thus if $\{x\}_1(x)$ is defined, then $range(\{f(x)\}_1) = \{0,1\} \not\subseteq ODD$. 
		\item But if $\{x\}_1(x)$ is undefined then $range(\{f(x)\}_1) = \{1\} \subseteq ODD$. 
		\item Hence, $A$ is not r.e.
	\end{itemize}
	% $A^c is r.e$
	\begin{itemize}[label = {}]
	\item Now we we want to show that $A^c$ is is r.e.
		\[A^c = \{x|  range(\{x\}_1) \not\subseteq ODD\}\]
	\item Notice that $x \in A^c$ iff there is some input $u$ and some $v$ such that $v$ codes a halting computation of program $\{x\}$ on input $u$, and $U(v)$ is not a odd. Using the T-predicate, we have,
		\[x \in A^c \leftrightarrow \exists u \ \exists v \ [T(x,u,v) \wedge  EVEN(U(v))] \]
		\[ EVEN(x) = \exists y\leq x \ \ 2y = x\]
	\item using a pairing function  to combine both existential quantifiers into one quantifier,
		\[ x \in A^c \leftrightarrow \exists z \ [T(x,K(z),L(z)) \wedge EVEN( U(L(z)) )]\]
	\item We get the form $x \in A^c \leftrightarrow \exists z \ R(x,z)$ where $R$ is recursive. Hence $A^c$ is r.e. 
	\end{itemize}
	It follows that $A$ is not recursive, and hence $A^c$ is not recursive. It also follows that $A$ is not r.e., because then  $A$ would be recursive. \\
\qed
% question 3
\item \proc{Solution:}
	\begin{itemize}[label ={}]
		\item Let $R(x,y)$ be a recursive relation, define $A = \{x|f(x) \ is \ defined\}$, where $f(x) = \mu y R(x,y)$.
		\item Then we have,
			 \[x \in A \iff f(x) \ is \ defined \iff  \exists y \ R(x,y)\]
		\item	 Clearly $f$ is computable. Suppose some program $\{e\}$ computes $f$, then using KNFT we get,
			\[f(x) = \{e\}_1(x) = U(\mu z T(e,x,z))\]
		\item Then clearly $ f(x) \ is \ defined \iff  \exists z \ T(e,x,z)$
		\item let $S(x,z) = T(e,x,z)$, clearly $S(x,z) $ is primative recursive and $ \exists y \ R(x,y) \iff  \exists z \ T(e,x,z)$.
	\end{itemize}
\qed
% question 4
\item \proc{Proof:}
	\begin{itemize}[label ={}]
		\item Now consider the case $\exists \ \leq$, say $A $ is $\exists x\leq t \ B(x)$, and this is in \textbf{TA}. Since this is a sentence, and by definition of  $\exists x\leq t$, we know $x$ cannot occur in $t$, it follows that $t$ is a closed term. Thus by Lemma A, \textbf{RA} can prove $t = s_n$ for some $n$.
		\item Now we do a induction on $n$
			\begin{itemize}[label ={}]
				\item  Base Case: n=0, then we have $\exists x\leq 0 \ B(x)$. By axiom $P7$ we have
				$x = 0$, then $\exists x\leq 0 \ B(x)$ is in \textbf{TA} means $B(0)$ is true. So by the induction hypothesis $B(0)$ is in \textbf{$RA_\leq$}. 
				\item
				\item Induction Hypothesis: for some $k$, $\exists x\leq s_k \ B(x)$ is in $TA$ then is also in \textbf{$RA_\leq$}.
				\item
				\item Inductive Step: $\exists x\leq s(s_k) \ B(x)$ is in \textbf{TA} . By $P8$ we know that 
				$x\leq s(s_k)  \supset (x \leq s_k \vee x= s(s_k))$.
				\item Hence we have,
					\[ \exists x\leq s(s_k) \ B(x) \iff  \exists x\leq s_k \ B(x) \vee B(s(s_k)) \]
				\item Then $\exists x\leq s(s_k) \ B(x)$ is in \textbf{TA} means $\exists x\leq s_k \ B(x)$ is in \textbf{TA} or $ B(S(s_k))$ is in \textbf{TA}.
				By induction hypothesis, we know $\exists x\leq s(s_k) \ B(x)$ is in \textbf{$RA_\leq$}. 
			\end{itemize}
		Therefore,  if $\exists x\leq t \ B(x)$ is in \textbf{TA} then it is also in  \textbf{$RA_\leq$}. 
	\end{itemize}
	\qed
% question 5
\item \proc{Proof:}
	In order to show that $\forall x \forall y\ \ x+y = y+x$, we first show $\forall x  \ \ 0+x = x$ and $\forall x \forall y \ \ sx+y = s(x+y)$, then using them to prove $\forall x \forall y \ \ x+y = y+x$.
	\begin{itemize}[label ={}]
	% part 1
	\item  First let's show $\forall x  \ \ 0+x = x$, we call this property $A1$.
			\begin{itemize}[label ={}]
			\item \[ f(x) = 0+x = x\]
			\item We use the induction axiom $Ind(f(x))$.
			\item Basis: x = 0
						\[ 0+0 = 0 \hskip2pc P3 \]
			\item Inductive Step: $x \leftarrow sx$
	
					\begin{align} \nonumber
                                        			0+sx &= s(0+x)  \hskip2pc  P4\\ \nonumber
                                        			& = sx \hskip4pc  Inductive \ Hypothesis \nonumber
                        			\end{align}
			\item Thus by $Ind(f(x))$, it follows that 
					\[ \textbf{PA}  \vdash \forall x\  f(x) \]
			\end{itemize}
	% part 2
	\item
	\item  Then let's show $\forall x \forall y \ \ sx+y = s(x+y)$ and we call this property $A2$.
			\begin{itemize}[label ={}]
			\item \[ g(y) = sx+y = s(x+y)\]
			\item We use the induction axiom $Ind(g(y))$.
			\item Basis: y = 0
					\begin{align} \nonumber
                                        			sx+ 0 &= sx \hskip3pc  P3\\ \nonumber
                                        				 & = s(x+0)	    \hskip1pc  P3 \nonumber
                        			\end{align}
			\item Inductive Step: $y \leftarrow sy$
	
					\begin{align} \nonumber
                                        			sx+sy &= s(sx+y)  \hskip3pc  P4\\ \nonumber
								&= s(s(x+y))	\hskip2.5pc   Inductive \ Hypothesis \\ \nonumber
                                        				& = s(x+sy)) 	\hskip3pc  P4 \nonumber
                        			\end{align}
			\item Thus by $Ind(g(y))$, it follows that 
					\[ \textbf{PA}  \vdash \forall x \ \forall y \  g(y) \]
			\end{itemize}
	% final
	\item
	\item  Finally let's show $\forall x \forall y \ \ x+y = y+x$
			\begin{itemize}[label ={}]
			\item \[ A(y) =  x+y = y+x\]
			\item We use the induction axiom $Ind(A(y))$.
			\item Basis: y = 0
					\begin{align} \nonumber
                                        			x+ 0 &= x \hskip3pc  P3\\ \nonumber
                                        				 & = 0+x	    \hskip1.5pc  A1 \nonumber
                        			\end{align}
			\item Inductive Step: $y \leftarrow sy$
	
					\begin{align} \nonumber
                                        			x+ sy &= s(x+y)  \hskip3pc  P4\\ \nonumber
								&= s(y+x)\hskip3pc   Inductive \ Hypothesis \\ \nonumber
                                        				& = sy+x 	\hskip3.5pc  A2 \nonumber
                        			\end{align}
			\item Thus by $Ind(A(y))$, it follows that 
					\[ \textbf{PA}  \vdash \forall x \ \forall y \  A(y) \]
			\end{itemize}
	\end{itemize}
	\qed
\end{enumerate}

\end{document}