\RequirePackage[l2tabu,orthodox]{nag}  % warn about common LaTeX pitfalls
\RequirePackage[ascii]{inputenc}  % input is 7-bit ASCII
\RequirePackage{fixltx2e}  % fix LaTeX2e kernel bugs

\documentclass[11pt,twoside]{article}
\usepackage{color}
\usepackage{amsthm}
\usepackage{amsmath}

\makeatletter
\newcommand{\dotminus}{\mathbin{\text{\@dotminus}}}

\newcommand{\@dotminus}{%
  \ooalign{\hidewidth\raise1ex\hbox{.}\hidewidth\cr$\m@th-$\cr}%
}
\makeatother
\usepackage{graphicx}
\graphicspath{ {image/} }
\usepackage{calc}  % arithmetic in length parameters
\usepackage{enumitem}  % more control over list formatting
\usepackage{fancyhdr}  % simpler headers and footers
\usepackage[margin=1in]{geometry}  % page layout
\usepackage{lastpage}  % for last page number
\usepackage{relsize}  % easier font size changes
\usepackage[normalem]{ulem}  % smarter underlining
\usepackage{url}  % verb-like typesetting of URLs
\usepackage{xfrac}  % nicer looking simple fractions for text and math
\usepackage{longtable}
\usepackage{tikz}
\usepackage{array}
\usepackage{tikz-timing}
\usetikzlibrary{arrows, shapes, backgrounds,fit}
\usepackage{tkz-graph}
% Set up fonts.
\usepackage[T1]{fontenc}  % use true 8-bit fonts
\usepackage{slantsc}  % allow slanted small-caps
\usepackage{microtype}  % perform various font optimizations
% Use Palatino-based monospace instead of kpfonts' default.
%\usepackage{newpxtext}
\ttfamily
\DeclareFontShape{T1}{\ttdefault}{m}{scsl}{<->ssub*\ttdefault/m/sc}{}
\DeclareFontShape{T1}{\ttdefault}{b}{scsl}{<->ssub*\ttdefault/b/sc}{}
% "Kepler" fonts.
\usepackage[nott,notextcomp]{kpfonts}
% Use curvier Latin Modern brackets instead of kpfonts' glyphs.
\DeclareSymbolFont{lmsymb}     {OMS}{lmsy}{m}{n}
\DeclareSymbolFont{lmlargesymb}{OMX}{lmex}{m}{n}
\DeclareMathDelimiter{\rbrace}{\mathclose}{lmsymb}{"67}{lmlargesymb}{"09}
\DeclareMathDelimiter{\lbrace}{\mathopen}{lmsymb}{"66}{lmlargesymb}{"08}

% Page layout: stretch text to fill up page.
\addtolength\footskip{.25\headheight}
\flushbottom

% Common list settings.

% Common macros.
%%  Common macros for the course CSC263H1 at the University of Toronto.
%%
%%  Copyright (c) 2014 Francois Pitt <fpitt@cs.utoronto.ca>
%%  last updated at 17:53 (EDT) on Sun 19 Oct 2014
%%
%%  CC BY-SA 4.0
%%  This work (the current file named 'macros-263.tex') is licensed under
%%  the Creative Commons Attribution-ShareAlike 4.0 International License.
%%  To view a copy of this license, visit
%%      http://creativecommons.org/licenses/by-sa/4.0/
%%  or send a letter to: Creative Commons, 444 Castro Street, Suite 900,
%%  Mountain View, California, 94041, USA.
%%  This is a human-readable summary of (and not a substitute for) the
%%  license.
%%  You are free to:
%%      Share -- copy and redistribute the material in any medium or format
%%      Adapt -- remix, transform, and build upon the material for any
%%          purpose, even commercially.
%%      The licensor cannot revoke these freedoms as long as you follow the
%%          license terms.
%%  Under the following terms:
%%      Attribution -- You must give appropriate credit, provide a link to
%%          the license, and indicate if changes were made. You may do so in
%%          any reasonable manner, but not in any way that suggests the
%%          licensor endorses you or your use.
%%      ShareAlike -- If you remix, transform, or build upon the material,
%%          you must distribute your contributions under the same license as
%%          the original.
%%      No additional restrictions -- You may not apply legal terms or
%%          technological measures that legally restrict others from doing
%%          anything the license permits.
%%  Notices:
%%      You do not have to comply with the license for elements of the
%%      material in the public domain or where your use is permitted by an
%%      applicable exception or limitation.
%%      No warranties are given. The license may not give you all of the
%%      permissions necessary for your intended use. For example, other
%%      rights such as publicity, privacy, or moral rights may limit how you
%%      use the material.

% Redefine \today in the style "DD Month YYYY".
\renewcommand*\today
 {\number\day\space\ifcase\month\or January\or February\or March\or
  April\or May\or June\or July\or August\or September\or October\or
  November\or December\fi\space\number\year}

% TeX trick so that math symbols are bold when text is.
\let\seiresfb\bfseries\def\bfseries{\boldmath\seiresfb}
\let\seiresdm\mdseries\def\mdseries{\unboldmath\seiresdm}

% Centered version of \llap and \rlap.
\providecommand*\clap[1]{\hbox to 0pt{\hss#1\hss}}

% Spacing macros with default argument.
\providecommand*\vfillstretch[1][1]{\vspace*{\stretch{#1}}}
\providecommand*\hfillstretch[1][1]{\hspace*{\stretch{#1}}}

% Abbreviations of latin phrases.
\let\latinabb\empty  % no special formatting
\providecommand*\ie{\latinabb{i.e.}}
\providecommand*\eg{\latinabb{e.g.}}
\providecommand*\etc{\latinabb{etc}}
\providecommand*\vs{\latinabb{vs}}

% General fonts and characters.
\let\longemph\textsl  % alternate way to emphasize
\let\strong\textbf  % strong emphasis
\let\code\texttt  % code fragments
\let\var\textsl  % multi-letter variables
\let\pred\mathbf  % predicates
\providecommand*\stup{\textsuperscript{st}}
\providecommand*\ndup{\textsuperscript{nd}}
\providecommand*\rdup{\textsuperscript{rd}}
\providecommand*\thup{\textsuperscript{th}}
\newcommand*\bbmath[1]{\ensuremath{\mathbb{#1}}}  % requires amssymb
\providecommand*\N{\bbmath{N}}  % requires amssymb
\providecommand*\Z{\bbmath{Z}}  % requires amssymb
\providecommand*\Q{\bbmath{Q}}  % requires amssymb
\providecommand*\R{\bbmath{R}}  % requires amssymb
\providecommand*\bigOh{\mathcal{O}}
\providecommand*\onehalf{\ensuremath{{}^1\!\!/\!_2}}
\providecommand*\letlinebreak{\penalty\exhyphenpenalty}
\providecommand*\halfthinspace{\kern.083333em}
\providecommand*\dash
   {\letlinebreak\halfthinspace---\halfthinspace\letlinebreak}
\let\per\slash  % for convenience

% For defined terms -- requires package ulem.
\newcommand*\defn[1]{\uline{\textit{#1}}}
\setlength{\ULdepth}{.2ex}

% General math macros.
\newcommand*\hiderel[1]{\mathrel{\hphantom{#1}}}
\providecommand*\comp[1]{\overline{#1}}  % set complement
\providecommand*\emptystr{\varepsilon}  % empty string
\providecommand*\lxor{\mathbin\oplus}  % exclusive or
\providecommand*\cat{\cdot}  % string concatenation
% Floors and ceilings, with optional size-changing command, e.g.,
% \floor[\Big]{x/2} becomes '\Big\lfloor{x/2}\Big\rfloor'.
\providecommand*\floor[2][]{{#1\lfloor}{#2}{#1\rfloor}}
\providecommand*\ceil[2][]{{#1\lceil}{#2}{#1\rceil}}
\providecommand*\lcm{\operatorname{lcm}}  % requires amsmath
\providecommand*\size{\operatorname{size}}  % requires amsmath
\providecommand*\len{\operatorname{len}}  % requires amsmath
\providecommand*\divides{\mathrel{|}}

% Redefined symbols from amssymb.
\renewcommand*\emptyset{\varnothing}  % rounder than default
\let\bigiff\iff
\renewcommand*\iff{\mathrel{\Leftrightarrow}}  % smaller than default
\let\bigimplies\implies
\renewcommand*\implies{\mathrel{\Rightarrow}}  % smaller than default
\renewcommand*\ge{\geqslant}  % with slanted line under the > sign
\renewcommand*\le{\leqslant}  % with slanted line under the < sign

% For algorithms, using either CLRS-style or Python-style pseudocode.
\let\ADT\textsc  % ADT names
\let\proc\textsc  % function names
\let\const\textsc  % constants (True, False, etc.)
\let\kw\textbf  % keywords (if, while, etc.)
\providecommand*\comm[1]{\textsl{\#\space#1}}  % comments
\providecommand*\opgets[1]{\mathrel{#1=}}
\providecommand*\eq{\mathrel{==}}
\providecommand*\True{\const{True}}
\providecommand*\False{\const{False}}
\providecommand*\None{\const{None}}
\providecommand*\cmod{\mathbin{\%}}
\providecommand*\nil{\const{nil}}

% Checkboxes and checklists.
% \checkmark already defined in amssymb
\makeatletter
\@ifpackageloaded{kpfonts}
   {\newcommand*\exmark{$\times$}}
   {\newcommand*\exmark{{\boldmath$\times$}}}
\makeatother
\newlength\exmarksize
\newlength\exmarkdepth
\newcommand*\checkbox[1][]
   {\settoheight\exmarksize{\exmark}
    \settodepth\exmarkdepth{\exmark}
    \addtolength\exmarksize{-\exmarkdepth}
    {\setlength\fboxrule{.1ex}\setlength\fboxsep{.1ex}%
    \fbox{\rule{0pt}{\exmarksize}\makebox[\exmarksize]{\smash{#1}}}}}
\newcommand*\checkedbox{\checkbox[\raisebox{.1ex}{\kern.2em$\checkmark$}]}
\newcommand*\exedbox{\checkbox[\exmark]}
\newenvironment*{checklist}{\begin{list}{\checkbox}{}}{\end{list}}
\newcommand*\checkeditem{\item[\checkedbox]}
\newcommand*\exeditem{\item[\exedbox]}
\newcommand*\boxeditem[1][]{\item[{\checkbox[#1]}]}

% Macros for drawing "underline" rules and half-boxes.
\providecommand*\urule[2][.5pt]{\rule[-.4ex]{#2}{#1}} % "underline" rule
% Underlined text box (with optional positioning argument).
\providecommand*\ubox[3][c]{\rlap{\urule{#2}}\makebox[#2][#1]{#3}}
% Underlined text box with caption (and optional caption positioning).
\providecommand*\capbox[5][c]{\ifx\empty#2\empty
    \else\rlap{\raisebox{-\baselineskip}{\makebox[#3][#1]{#2}}}\fi
    \ubox[#4]{#3}{#5}}

% For student numbers.
\providecommand*\hb{\urule{1.2em}}      % horizontal bar
\providecommand*\vb{\urule[1ex]{.5pt}}  % vertical bar
\providecommand*\studentnumberboxes     % now 10-digits long!
   {\vb\hb\vb\hb\vb\hb\vb\hb\vb\hb\vb\hb\vb\hb\vb\hb\vb\hb\vb\hb\vb}
\newlength\numberboxwidth
\settowidth\numberboxwidth{\studentnumberboxes}

% Command to format sample solutions for tutorials.
\newcommand\samplesolution[1]
   {\ifsolutions\begin{quote}\sffamily{#1}\end{quote}\fi}

% Marking-related macros.
\providecommand*\markitem[2][]
   {\item{\bfseries#2}\ifx\empty#1\empty\else
    \space[$#1$ mark\ifnum#1>1 s\fi]\fi:\quad\ignorespaces}
\providecommand*\errorcode[2][]
   {\item{\bfseries error code #2}\ifx\empty#1\empty\else
    \space[#1]\fi:\quad\ignorespaces}
\providecommand*\commonerror[1][]
   {\item{\bfseries common error}\ifx\empty#1\empty\else
    \space[#1]\fi:\quad\ignorespaces}
\makeatletter
\providecommand*\heading[1]%
   {\@startsection{heading}{9}{0pt}{-.75ex plus -1.5ex minus -.5ex}
    {-.5em plus -.5em minus -.25em}{\normalfont}*{\textsc{#1}}\mbox{}}
\makeatother
\newcommand*\st{\mathrel{|}}  % "such that" for set extension

% Headings.
\pagestyle{fancy}
\let\headrule\empty
\let\footrule\empty
\lhead{CSC\,418\,H1}
\chead{\large\scshape Assignment \#\,3}
\rhead{\scshape Fall 2015}
\lfoot{\scshape Dept.\@ of Computer Science, University of Toronto,
       St.~George Campus}
\cfoot{}
\rfoot{\scshape page \thepage\space of \pageref{LastPage}}


\begin{document}
\[Rui \ Ji \ (1000340918)\]
\begin{enumerate}[leftmargin=0pt]
% question 1
\item Following is a RM program which computes the function $f(x) = 2x$, suppose $x$ is stored at $R_1$.
	\begin{itemize}[label = {}]
		\item $c_0: R_2 \leftarrow 0 \ \ \ \ \ \ \ \ \ \ \ \ \ \ \ \ \ \ \ \ \ \ \ \ \ Z_2$
		\item $c_1: goto \ 6 \ if \ R_1 = R_2  \ \ \ \ \ \ \ J_{1,2,6}$
		\item $c_2: R_1 \leftarrow R_1+1  \ \ \ \ \ \ \ \ \ \ \ \ \ \ \ \ \ S_{1}$
		\item $c_3: R_2 \leftarrow R_2+1  \ \ \ \ \ \ \ \ \ \ \ \ \ \ \ \ \  S_{2}$
		\item $c_4: R_2 \leftarrow R_2+1  \ \ \ \ \ \ \ \ \ \ \ \ \ \ \ \ \  S_{2}$
		\item $c_5: goto \ 1 \ if \ R_1 = R_1  \ \ \ \ \ \ \ J_{1,1,1}$
		\item $c_6$
	\end{itemize}
	First we initialize $R_2$ to be $0$. Command $c_1$ to $c_5$ is a loop, within the loop we add $2$ to $R_2$ each time and add $1$ $R_1$, and we will return $R_1$ when $R_1 = R_2$. \\
	Note that suppose after $n_{th}$ iteration $R_1 = R_2$, which means $n+x = 2n$ then $n=x \Rightarrow R_1 = 2x$.\\
	\qed
% question 2
\item
\begin{enumerate}
\item Since limited subtraction, bounded sum, bounded products  and divisibility are all  primitive recursive,
	\[Bit(x,i) = 2^{i+1} |( x \dotminus \sum_{t<i} 2^t B(x,t))\]
	Note 
	\[2^{i+1} = h(2, i+1) = \prod_{z<i+1}2\]
	\[2^{t} = h(2, t) = \prod_{z<t}2\]
	 then $Bit(x,i)$ is also primitive recursive. \\
	Following is an example computes $Bit(6,0), Bit(6,1), Bit(6,2)$,
	\begin{itemize}[label ={}]
	 \item \[Bit(6,0) = 2^{0+1} |( 6 \dotminus \sum_{t<0} 2^t B(6,t))\]
	  by definition of bound sum we know that 
	  			\[\sum_{t<0} 2^t B(6,t)=0\]
	Then 
				\[Bit(6,0) = 2|(6-0) = 0\]
	\item 
	\item  \[Bit(6,1) = 2^{1+1} |( 6 \dotminus \sum_{t<1} 2^t B(6,t))\]
	 by definition of bound sum we know that 
	  			\[\sum_{t<1} 2^t B(6,t)=B(6,0) = 0\]
	Then 
				\[Bit(6,1) = 4|(6-0) = 1\]
	\item 
	\item  \[Bit(6,2) = 2^{2+1} |( 6 \dotminus \sum_{t<2} 2^t B(6,t))\]
	 by definition of bound sum we know that 
	  			\[\sum_{t<1} 2^t B(6,t)=2^1B(6,1) = 2\]
	Then 
				\[Bit(6,2) = 8|(6-2) = 0\]
	 \end{itemize}
\item Suppose $x$ in binary has $n$ bits, we know that $2^{(x-1)} \geq x$ for $( x \in N \wedge 1 \leq x)$, then
$i$, where $i<x$ cover all $n$ bits of $x$.\\
	Since bounded sum and $Bit(x,i)$ is primitive recursive,
 	 \[NumOnes(x) =  \sum_{i<x} B(x,i)\]
	 then $NumOnes(x)$ is also primitive recursive. 
\end{enumerate}
\qed
% question 3
\item
\begin{enumerate}
\item $\proc {Claim:}$ $A$ is neither recursive nor r.e. $ A^c$ is r.e but not recursive.
	\begin{itemize}[label = {}]
	% $A^c not  r.e$
	\item First we show that $A$ is not r.e. Notice, it suffices to show that
			\[K^c \leq_m A\]
		Thus we want a total computable function $f(x)$ such that
			\[x \in K^c \iff f(x) \in A \]
		i.e we want
			\[ \{x\}_1(x) = \infty \iff dom(\{f(x)\}_1) \subseteq PRIMES\]
		We can define $f(x)$ implicitly using the S-m-n Theorem as follows:
			\begin{gather*}
			\{f(x)\}_1(y) = 
				\begin{cases}
				\{x\}_1(x)  & \text{if } y \neq 2\\
				y  & \text{if } y = 2
				\end{cases}
			\end{gather*}
		\item Thus if $\{x\}_1(x)$ is defined, then $\{f(x)\}_1(y) $ is only defined for all $y \in N$, so $dom(\{f(x)\}_1) \not\subseteq PRIMES$. 
		\item But if $\{x\}_1(x)$ is undefined then $\{f(x)\}_1(y) $ is undefined for only for $y=2$; hence,  $dom(\{f(x)\}_1) \subseteq PRIMES$. 
		\item
	\end{itemize}
	% $A^c is r.e$
	\begin{itemize}[label = {}]
	\item Now we we want to show that $A^c$ is is r.e.
		\[A^c = \{x|  dom(\{x\}_1) \not\subseteq PRIMES\}\]
	\item Notice that $x \in A^c$ iff there is some input $u$ and some $v$ such that $v$ codes a halting computation of program $\{x\}$ on input $u$, and $u$ is not a prime. Using the T-predicate, we have,
		\[x \in A^c \leftrightarrow \exists u \ \exists v \ [T(x,u,v) \wedge \neg Prime(u)] \]
	\item using a pairing function  to combine both existential quantifiers into one quantifier,
		\[ x \in A^c \leftrightarrow \exists z \ [T(x,K(z),L(z)) \wedge \neg Prime(K(z))]\]
	\item We get the form $x \in A^c \leftrightarrow \exists z \ R(x,z)$ where $R$ is recursive. Hence $A^c$ is r.e. 
	\end{itemize}
	It follows that $A$ is not recursive, and hence $A^c$ is not recursive. It also follows that $A$ is not r.e., because otherwise $A$ would be recursive. 

\item $\proc {Claim:}$  $B$ is r.e but not recursive, $B^c$ is neither recursive nor r.e.
	\begin{itemize}[label = {}]
	\item First we show that $B$ is  r.e. Let's define following function $UP(z,x)$
	\item  \[UP(z,x)= U(min \ y < (A(z_0,x)+ z)   \ T(z_{1}, x, y))\]
        		\begin{itemize}
        		\item	 It is clear that $UP(z,x)$ is a total function, since $U,min, A,T$ are all total function.
        		\item Also for each $e \in N$, the unary function $UP(e,x)$ is primitive recursive, since $U, A,T$ are primitive recursive function and $min \ y < (A(z_0,x)+ z)$ is bounded for each $x$; hence, $UP(e,x)$ is primitive recursive.
        		\item For each unary primitive recursive function $f(x)$, there exist $Ackermann's\ Function$ $A_n(x)$, such that $A(n,x) + B > f(x)$; Hence,  exist  some $k$ such that $A(n,x) + B > Comp_{\proc P}(x)$, where ${\proc P}$ computes $f$.
        	
        			\begin{itemize}[label={}]
        			\item Every primitive recursive function $f(\vec x)$ is computable by a RM program $\proc P$ such that the function $Comp_{\proc P}(\vec x)$ is primitive recursive, where
        		\[ Comp_{\proc P} (\vec x) \ is \ the \ number \ coding \ the \ computation \ of \proc \ P \ on \ input\ \vec x\]
        			\end{itemize}
			using KNFT we have,	
					\[f(x) =  U(min \ y < (A(k,x)+ B)   \ T(\#\proc P, x, y)) \]		
			let $z_0 = max(A,B)$, $z_1 = \#\proc P$ and $z = 2^{z_0}2^{z_1}$, we have
					\[f(x) = U(min \ y < (A(z_0,x)+ z)   \ T(z_{1}, x, y))\]
			hence, for each unary primitive recursive function $f(x)$, there exist some $e \in N$ such that $f(x) = UP(e,x)$.
		\end{itemize}
		Then we have $B = ran(UP)$, which means B is r.e.
	
	\end{itemize}
	\begin{itemize}[label ={}]
	\item Now we show that $B$ is  not recursive. 
	\item Intuitively $B \neq N$, then $B$ is not recursive
	It follows that $B$ is not recursive, and hence $B^c$ is not recursive. It also follows that $B^c$ is not r.e., because otherwise $B$ would be recursive. 
	\end{itemize}
\end{enumerate}
% question 4
\item
\begin{enumerate}
\item \begin{itemize}[label = {}]
	\item Let $\#\proc{p}$ be the number encoding the RM program $\proc{p}$.
	\item
	\item $STATE_\proc{p}(\vec x,0 ) = g(\vec x) = p_0^0p_1^{x-1}...p_n^{x_n}$
	\item $STATE_\proc{p}(\vec x,t+1) = Nex(STATE_\proc{p}(\vec x,t), \#\proc{p})$
	\item
	\item Note $g(\vec x)$ is primitive recursive; hence, $g(\vec x) \in \varepsilon$, also we assuming that $Nex(u,z) \in \varepsilon$, where $Nex(u,z) = u$ if $u$ is the halting states,  then we can get $STATE_\proc{p}(\vec x,t )$ is also in $\varepsilon$.
	\item
	\end{itemize}
\item \begin{itemize}[label = {}]
	\item For any $f \in \mathcal{C}$, suppose program $p$ computes f, by definition we know that there exist $k \in N$ such that $Time_\proc{p}(\vec x) \leq E_k(x_1+x_2+...+x_n)$, which means $f(x)$ will halt within $E_k(x_1+x_2+...+x_n)$ steps.
	\item Using KNFT we can get $f(\vec x) = U(min \ y<E_k(x_1+x_2+...+x_n)\ \   T(\#\proc{p}, \vec x, STATE_\proc{p}(\vec x,y)))$, where$T(x,y,z)$ is Kleene $T$ predicate.
	\item Since $U, T, min$ are all primitive recursive; hence, $U, T, min \in \varepsilon$, also we know that 
	$E_k \in \varepsilon$; therefore, $f \in \varepsilon$. Then we get $\mathcal C \in \varepsilon$.
	\end{itemize}
	\qed
\end{enumerate}


\end{enumerate}

\end{document}